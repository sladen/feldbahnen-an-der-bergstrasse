% -*- coding: utf-8; mode: latex -*-
% twoside
\documentclass[a4paper]{report}
\usepackage[utf8]{inputenc}

\bibliographystyle{unsrt}

\usepackage[british]{babel}
\usepackage[babel=true]{microtype}
\usepackage{natbib}
\setlength{\bibsep}{0.0pt}
\usepackage[
  pdfauthor={Paul Sladen},
  pdfencoding=unicode,
  backref=section
]{hyperref}
\hypersetup{
  colorlinks   = true,
  urlcolor     = blue,
  linkcolor    = blue,
  citecolor    = red,
  pdfborder = {0 0 0}
}
\usepackage[english]{isodate}
\begin{document}
\title{Feldbahnen an der Bergstraße und bei Wiesloch\\Narrow gauge railways along the~Bergstrasse and around~Wiesloch}
\author{Paul Sladen}
\date{\textsc{draft}}

\maketitle
\tableofcontents{}

%\begin{frontmatter}
\part{Foreward / Vorwärt}
\section*{Abstract}

In south-west Germany the Upper Rhine valley from a gorge to a expands
to a wide plain.  Nestling at the foot of the eastern-edge runs the
Bergstrasse (Mountain Road) route, an perfect north-south line between
Darmstadt in Hesse, through Heidelberg, to Wiesloch in
Baden-Württemberg. Hidden fault lines the sunken valley cause
high-quality clay and mineral deposits to rise to surface of the
valley, while the towering cliff edge has been punctuated with
numerous mines and quarries; each with its narrow gauge railway
system and trains: {\it the Feldbahnen an der Bergstrasse.}

\section*{Structure}

Research and collection of material dates from 2016-- onwards.
Citations are in German, with the write-up in English.

The principal corridor of interest is the linear strip enclosed
between the Frankfurt--Heidelberg--Karlsruhe mainline railway, running
on the floor of the Upper Rhine valley, and the tall escarpment just
beyond---the Bergstrasse route.  A large cluster of activity is
identified in the vicinity of Wiesloch, a historic centre of mining,
at the southern end.  At the foci of this cluster was created the
open-air Wiesloch narrow gauge and industrial museum.

It is argued that the unique structure of the mining industry and
availability of material at Wiesloch enabled the museum, in the same
way that the unique geological structure of the Wiesloch cluster
enabled the mining activity in the first place.

Other narrow gauge systems encountered are covered in the appendix, 

%\end{section}
%\end{frontmatter}

\part{Feldbahnen an~der~Bergstrasse}

\chapter{Industrial}
\section{Darmstadt}

The first 18 kilometres south from Darmstadt to Alsbach carry the
metre-gauge tram tracks of the Hessian Railway Company (HEAG;
Hessische Eisenbahn-Aktiengesellschaft).  Since summer 1997 the steam
tram "Feuriger Elias" and wagons run along the Bergstrasse, operated
by the Arbeitsgemeinschaft Historische HEAG-Fahrzeuge.

\section{Weinheim}

Metre gauge tracks once again join the Bergstrasse for 25 km to
Leimen, and previously 30 km to Wiesloch.  Above the Weinheim mainline
railway station, the Upper Rhine Railway Company (OEG) turns south to
the Old Weinheim OEG station.  Here the OEG "Tram 5" runs on a
circular loop via Heidelberg, Mannheim, Viernheim and back to Weinheim
taking 2 hours 20 minutes.  A slight figure-of-eight occurs in
Mannheim with the route crossing over itself to reach Mannheim Hauptbahnhof.

\subsection{Porphyrwerke Weinheim}

Weinheim's Wachsenburg Castle obscures the Weinheim mega quarry.  One
Jung locomotive is preserved in Dossenheim 10 kilometres further south.

\subsection{Mannheimer Strasse}

Immediately next to the west ramp of the Mannheimer Strasse bridge, in
the ground lay two sets of narrow gauge tracks heading under
factory building door.

\section{Viernheim}

Viernheim is back over the border in Hesse.

\subsection{Chemische Fabrik Fridlingen}

Between 1957--2003 the Fridlingen Chemical Factory uses 600 mm gauge
flat wagons shunted by a Diema DL6 locomotive (2086/1957).  The DL6
was donated to the Wiesloch Feldbahnmuseum and restored, with the flat
wagons forming the construction train, and Lego/VIP wagon.

\section{Großsachsen}

Most of the OEG route was double tracked in 2015, but between the
Großsachsen OEG and Großsachsen South tram stops, the OEG runs as
single track directly in Bergstrasse road itself, squeezing past
buildings and overhanging the pedestrian pavement.

\section{Dossenheim}

Two huge scars on the mountain cliff tower above the town: Vatter
Quarry on one side, and Leferenz Quarry on the otherside.  Dual gauge
1000 mm/1435 mm track served two goods stations at the bottom of the
valley, respectively the locations of the present-day Dossenheim North
and Dossenheim South tram stops.  A private Feldbahn collection
operates in the lower-level of the Leferenz quarry.

Standard gauge tracks ran along the present-day Alte Gütertrasse
footpath: during car park construction in 2017 for the Doctor's House
at the corner of Bahnhofstrasse, wooden sleepers from the tracks were
unearthed.  As dual-gauge the line continued to Dossenheim South,
before diverting from the Bergstrasse and running diagonally through
the middle of the Heidelberg University Neuenheimer Feld Campus (INF).
A single-track goods railway bridge with four rails crossed over the
River Neckar; with the bridge abutments and southern embankment still remaining.

\subsection{Roundabout}

In the middle of the roundabout where Aussiedlerhöfe joins the
Bergstrasse at the north edge of Dossenheim, a wagon is preserved on a
short length of track.  The area between the roundabout and Dossenheim
North station was previously the loading yard for the Vatter Quarry.

\subsection{Heimatmuseum}

The courtyard of the Dossenheim Local History contains a mining wagon and short length of track.

\subsection{Steinbruch Vatter}

Quarrying and removable of the hillside partly destroyed the
Schauenburg Castle over Dossenheim.  Following closure of the quarry, three narrow
gauge locomotives, along with sections of track, tipper wagons, and a
crane were transported to the Wiesloch Feldbahn museum.

\subsection{Steinbruch Leferenz}

\section{Heidelberg}

\subsection{Schleuse Heidelberg}

The Neckar river through the centre of Heidelberg is contained between
two sets of dams and locks.  A narrow gauge system was used during
construction of the Heidelberg ship lock slightly up-river next to
Heidelberg Altstadt station.

For porting of canoes around the weird, a hand-operating narrow gauge
track exists, along with a small carrier wagon.  This single track is
approximately 200metres long, with a narrow tunnel in the middle!

Similar tracks are present upstreams at Neckargemund, and Neckarstein locks.

\section{Leimen}

Heidelberg Cement has its research and development facilities between
Heidelberg-Rohrbach and Leimen.  The factory is served by the tram
stop "Zementwerk", industrial mainline sidings, a huge overhead
ropeway; and previous a rope-operated Feldbahn system.

\subsection{Steinbruch Leimen}

A dual track narrow gauge Feldbahn ran down from the cliff via an
incline and over across bridge over the Bergstrasse and 1000mm tram
tracks, into the top of the cement factory.

The 20-metre long and iconic Heidelberg Cement bridge was removed and
cut up in 2016, with the embankment removed afterwards.  The Feldbahn
previously crossed in the quarry were Oberklamweg road/cycle-track and
L100 road to Boxberg meet.

\section{Steinbruch Heidelberg Cement}

At the south-east of Nussloch is the huge Heidelberg Cement quarry,
with material transport originally on the tram tracks and later by a
5-kilometre long ropeway back to the Cement factory.

\chapter{Tourist}
\section{Dossenheim}
\subsection{Schilling Feldbahn}

Since 1987 Martin Schilling and family assembled a a large collection
of narrow gauge rolling stock.  The line is open to the public for
passenger rides on certain days of the year; with an push-pull route
operated down a slight gradient from the main display area to the
locomotive shed and back again.

\subsubsection{Quarry Feldbahn}

A small Feldbahn display exists under the care of the Town of
Dossenheim; including a cosmetic connection to the Schilling Feldbahn system.

\section{Heidelberg-Königstuhl}

\subsection{Märchenparadies}

At the top of the funicular railway running up to the Königstuhl,
there is a small them park containing the Parkbahn Märchenparadies
miniature railway.  The route operates in a clockwise circular
direction.

\part{The Wiesloch cluster}

\chapter{Historical}

\section{Nußloch}

Nussloch forms the northern edge of the tight Wiesloch cluster, here
the brickworks was placed far outside the town, with the clay pit only
1 kilometre from the Tonwaren Industrie Wiesloch workings.

\subsection{Dampfziegelei Nußloch}

Clay was mined in the Dammstücker pit, between the B3 road, and the
Leimbach river.  A small concrete and girder bridge carried the
Feldbahn trains over the River Leimbach, immediately down a ramp, and
through a circular tunnel under the new road between Nußloch and Walldorf
(K4256).  The brickworks were demolished, with the Racket Centre Nußloch
tennis club constructed on the site.

As of 2019 the Feldbahn bridge over the Leimbach remained in-situ, with the
girders that used to hold the rails visible, but without tracks.

\section{Wiesloch}

\subsection{Hessler Kalksteinbruch}

Narrow gauge wagons used to bring material from the pit to the factory in Alt-Wiesloch.

\subsection{Neuer Friedrichstollen}

Opposite the disused adit lays a locomotive shed.

\subsection{Tonwarenindustrie Wiesloch}

Barely 1 kilometre of the Dammstücker pit serving Nußloch, the much
bigger Dämmelwald pit served the Tonwarenindustrie Wiesloch brick
factory.

%\subsection{Wiesloch Feldbahn and Industrial Museum}
\chapter{Feldbahn- und Industriemuseum Wiesloch}

In 1999/2000 an interest-group contacted the Town authoirities about
rescuing the 1905-built narrow guage locomotive shed that had been
buried under the earth up to the roof top for 30 years, since 1979.

During construction of the Bundersstrasse 3 in 1979 the tracks of the Feldbahn were cut in half, and additionally the material from a large cutting was used to raise the land level by 3.5 metres over a wide area, effectively buring the locomotive shed.

Initially the town granted the group a lease for 80m * 20m immediately
surrounding the locomotive shed.  Several thousand tonnes of earth
were removed, uncovering locomotive to allow restoration and the
create of a pair of short mineshafts.

\section{Frauenweiler}

A planned settlement just beyond the south edge of Wiesloch.

Both sand and clay were excavated, subsequently turned into nature
parks. A Feldbahn route was used to remove the clay to the next town.

\subsection{Bausteinwerke Kälberer}

Sand mining firm.

\subsection{Bottloch}

A lake called Bottloch marks the source of the clay processed by Emil
Bott AG in Rauenberg.

\section{Rauenberg}

The clay from the Frauenweiler pits travelled several kilometres
before meeting the Wiesloch--Waldangeltalloch local standard gauge
railway.  The tracks then travelled side-by-side in parallel until
reaching the edge of Rauenberg.\cite{schubert1989}

In September 1967 a exhibition of three narrow gauge locomotives was
organised on this side-by-side section, with steam locomotive SWEG 14
hauling trains of visitors before being photographed side-by-side.

\section{Malsch}

\subsection{Ziegelwerk Malsch}

\part{Feldbahnen in der Nähe}

\chapter{Extras}

\section{Mannheim}

Mannheim is the industrial counterpart to the historic of Heidelberg,
nestled downstream where the Neckar joins the Rhine opposite
Ludwigshafen.  Two significant agents re-selling narrow gauge railway
equipment were based in Mannheim: with their dealer plates found on
large amounts of equipment.

\subsection{Martin Kallmann (Händler)}

\subsection{Feldbahnfabrik Breidenbach}

Feldbahnfabrik Breidenbach \& Company, was a large agent pre-ordering
and selling on Feldbahn locomotives and rolling stock.

\subsection{Technoseum}

The large interactive industrial museum in Mannheim has a couple of
narrow exhbitions, plus a long length of dual gauge 1000mm/1435mm through the centre of the building.

\subsubsection{Technoseum Feldbahn}

In the grounds of the museum, a circular tourist route operates for
visitors.  Trains runs anti-clockwise, with a small station and stub
siding where visiting locomotives can be exchanged for two-train
operation.  A side track runs down an incline into the lower logistics
yard of the museum, where a two-track shed exists for the rolling
stock and locomotives.  A short bridge is on the route near the main
museum entrance, after this the track layout was adjusted to reduce
some of the gradients and allow a more gentle route down to the
sidings.

\section{Mannheim-Friedrichsfeld}

Half way between Heidelberg and Mannheim is a sub-station belonging to
the Deutsche Bahn.  On the edge of this site is the Historischal
Eisenbahn Mannheim e.V.  At the rear of the buildings a
round-and-route Feldbahn system.

\section{Eppelheim}

\subsection{Modellbahn Schuhmann}

The Schuhmann model train shop contains various railway items in its yard.

\section{Speyer}

\subsection{Herrenteich}

On the opposite side of the river lays the Herrenteich pipe works.
This had a Feldbahn system for transporting clay, and then a series of
additional tracks leading into the ovens for firing.

\subsection{Technik Museum}

A narrow gauge locomotive is on the site.

\section{Sondernheim}

\subsection{Ziegelei Sondernheim}

The former internal Feldbahn of the brickworks running through the
drying sheds was converted into a long oval for passenger use.

\part{Appendix}

\bibliography{feldbahn}

\end{document}
